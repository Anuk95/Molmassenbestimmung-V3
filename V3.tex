\documentclass[12pt,a4paper,titlepage,headinclude,bibtotoc]{scrartcl}

%---- Allgemeine Layout Einstellungen ------------------------------------------

% Für Kopf und Fußzeilen, siehe auch KOMA-Skript Doku
\usepackage[komastyle]{scrpage2}
\pagestyle{plain}
\setheadsepline{0.5pt}[\color{black}]
\automark[section]{chapter}


%Einstellungen für Figuren- und Tabellenbeschriftungen
\setkomafont{captionlabel}{\sffamily\bfseries}
\setcapindent{0em}


%---- Weitere Pakete -----------------------------------------------------------
% Die Pakete sind alle in der TeX Live Distribution enthalten. Wichtige Adressen
% www.ctan.org, www.dante.de

% Sprachunterstützung
\usepackage[ngerman]{babel}

% Benutzung von Umlauten direkt im Text
% entweder "latin1" oder "utf8"
\usepackage[utf8]{inputenc}

% Pakete mit Mathesymbolen und zur Beseitigung von Schwächen der Mathe-Umgebung
\usepackage{latexsym,exscale,stmaryrd,amssymb,amsmath}


\usepackage[nointegrals]{wasysym}
\usepackage{eurosym}

% Anderes Literaturverzeichnisformat
%\usepackage[square,sort&compress]{natbib}
\usepackage{hyperref}
% Für Farbe
\usepackage{color}
\usepackage{graphicx}
\usepackage{wrapfig}
\usepackage{subfigure}

% Caption neben Abbildung
\usepackage{sidecap}


% Befehl für "Entspricht"-Zeichen
\newcommand{\corresponds}{\ensuremath{\mathrel{\widehat{=}}}}
% Befehl für Errorfunction
\newcommand{\erf}[1]{\text{ erf}\ensuremath{\left( #1 \right)}}


%Fußnoten zwingend auf diese Seite setzen
\interfootnotelinepenalty=1000

%Für chemische Formeln (von www.dante.de)
%% Anpassung an LaTeX(2e) von Bernd Raichle
\makeatletter
\DeclareRobustCommand{\chemical}[1]{%
  {\(\m@th
   \edef\resetfontdimens{\noexpand\)%
       \fontdimen16\textfont2=\the\fontdimen16\textfont2
       \fontdimen17\textfont2=\the\fontdimen17\textfont2\relax}%
   \fontdimen16\textfont2=2.7pt \fontdimen17\textfont2=2.7pt
   \mathrm{#1}%
   \resetfontdimens}}
\makeatother
\usepackage{textcomp}
\usepackage{upgreek}
%\begin{document}
%$\upmu$
%\end{document}
%Honecker-Kasten mit $$\shadowbox{$xxxx$}$$
\usepackage{fancybox}

%SI-Package
\usepackage{siunitx}

%keine Einrückung, wenn Latex doppelte Leerzeile
\parindent0pt

%Bibliography \bibliography{literatur} und \cite{gerthsen}
%\usepackage{cite}
\usepackage{babelbib}
\selectbiblanguage{ngerman}

\usepackage{siunitx}
%\begin{document}
 % \SI{1.55}{\micro\metre}
\sisetup{math-micro=\text{µ},text-micro=µ}
\begin{document}

\begin{titlepage}
\centering
\textsc{\Large Praktikum zur Einführung in die physikalische Chemie,\\[1.5ex] Universität Göttingen}

\vspace*{2cm}

\rule{\textwidth}{1pt}\\[0.5cm]
{\huge \bfseries
  V4: Molmassenbestimmung\\[1.5ex]
  nach Viktor Meyer}\\[0.5cm]
\rule{\textwidth}{1pt}

\vspace*{1cm}


\begin{Large}
\begin{tabular}{ll}
Durchführende: &  Alea Tokita, Julia Stachowiak\\
Assistentin: & Annemarie Kehl\\
 Versuchsdatum: & 21.12.2015\\
 Datum der ersten Abgabe: & 11.01.2016
\end{tabular}
\end{Large}

\vspace*{2cm}

\begin{Large}
\fbox{
  \begin{minipage}[t][5,5cm][t]{8cm} 
   Werte für Campher:\\
   $M_A = (13 \cdot 10 \pm 3 \cdot 10){~} \mathrm{g{~}mol^{-1}}$\\
   $M_B = (12 \cdot 10 \pm 2 \cdot 10){~} \mathrm{g{~}mol^{-1}}$\\\\
   Werte für Kaliumchlorid:\\
   $M_A = (6 \cdot 10 \pm 3 \cdot 10){~} \mathrm{g{~}mol^{-1}}$\\
   $M_B = (5\cdot 10 \pm 3\cdot 10 ){~} \mathrm{g{~}mol^{-1}}$\\
  
  \end{minipage}
}
\end{Large}

\end{titlepage}

\tableofcontents

\newpage
\section{Theorie}
In der Bestimmung der Molmasse nach Viktor Meyer wird die Molmasse einer leicht verdampfbaren Substanz mithilfe des von ihr verdrängten Gasvolumens beim Verdampfen bestimmt.



\section{Auswertung}








\section{Fehlerrechnung}

\subsection{absolute Fehler}

Die absoluten Fehler der Messgrößen ergeben sich aus der letzten Dezimalstelle. Der absolute Fehler des Athmosphärendrucks $p_\mathrm{B}$ errechnet sich aus dem Fehler der Kuppenhöhe und dem der Temperatur bei der Umrechnung in Torr. Die Fehler werden unten dargestellt.\\

\begin{table} [h]
\begin{tabular} {p {3 cm} p {5 cm}}
	$\Delta m $ & $=0,001$ g\\
	$\Delta p_\mathrm{B}  $ & $=0,07\, \mathrm{Torr} = 9,3325$ Pa\\
	$\Delta \delta_\mathrm{H_2O} $ & $=0,01\, \mathrm{kg} \cdot 				\mathrm{m^{-3}}$\\
	$\Delta r $ & $=0,01$\\
	$\Delta V $ & $= 0,1 \cdot 10^{-6} \mathrm{m^3}$\\
	$\Delta p_\mathrm{H_2O}(T_\mathrm{Z}) $ & $= 0,1$ Pa\\
	$\Delta h $ & $ =0.005$ m\\

\end{tabular}
\end{table}

\subsection{Fehlerfortpflanzung}

Nach der Gauß'schen Fehlerfortpflanzung

\begin{equation}
\Delta f = \sqrt{\sum_\mathrm{i} \left(\frac{d}{dx_\mathrm{i}}\right)^{2} \cdot \Delta x_\mathrm{i}^2}
\end{equation}


ergibt sich für den hydrostatischen Druck $\Delta p_\mathrm{hyd}$ folgender Fehler:\\

\begin{equation}
\Delta p_\mathrm{hyd} =  \sqrt{(g \cdot h)^2 \cdot \Delta \delta_\mathrm{H_2O}^2 + (\delta_\mathrm{H_2O} \cdot g)^2 \cdot \Delta h^2}
\end{equation}


Anschließend ergibt sich für den Fehler des Luftdruckes $\Delta p_\mathrm{L}$:\\


\begin{equation}
\Delta p_\mathrm{L} = \sqrt{\Delta p_\mathrm{B}^2 + \Delta p_\mathrm{hyd}^2 + p_\mathrm{H_2O}(T_\mathrm{Z})^2 \cdot \Delta r^2 + (1-r)^2 \cdot \Delta p_\mathrm{H_2O}(T_\mathrm{Z})}
\end{equation}

Und anschließend für den Fehler der Molaren Masse:\\

\begin{equation}
\Delta M = \sqrt{\left(\frac{R\cdot T}{p\cdot V}\right)^2 \cdot \Delta m^2 + \left(\frac{m\cdot R}{p\cdot V} \right) \cdot \Delta T^2 + \left(-\frac{m\cdot R\cdot T}{p^2 \cdot V}\right)^2 \cdot \Delta p^2 + \left(-\frac{m \cdot R \cdot T}{p \cdot V^2} \right)^2 \cdot \Delta V^2}
\end{equation}

Mit $p = p_\mathrm{L}$ ergeben sich damit folgende Fehler:\\

\begin{table} [h]
\centering
\begin{tabular}{|p{4 cm}||p{4 cm}|p{4 cm}|}
        \hline
		Fehler & Messung 1 & Messung 2\\
         \hline 
        $\Delta p_\mathrm{hyd}$ in Pa & $48,95$  & $48,95$ \
        \hline
        $\Delta p_\mathrm{L}   $ in Pa & $1344$  & $1344$ \\
        \hline
        $\Delta M $ in $\mathrm{g{~}mol^{-1}}$ & $1,72$ & $1,74$ \\
        \hline     
\end{tabular}
\end{table}

\subsection{Vergleich mit Literaturwerten}

\subsection{Diskussion systematischer Fehler}




\end{document}

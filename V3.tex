\documentclass[12pt,a4paper,titlepage,headinclude,bibtotoc]{scrartcl}

%---- Allgemeine Layout Einstellungen ------------------------------------------

% Für Kopf und Fußzeilen, siehe auch KOMA-Skript Doku
\usepackage[komastyle]{scrpage2}
\pagestyle{plain}
\setheadsepline{0.5pt}[\color{black}]
\automark[section]{chapter}


%Einstellungen für Figuren- und Tabellenbeschriftungen
\setkomafont{captionlabel}{\sffamily\bfseries}
\setcapindent{0em}


%---- Weitere Pakete -----------------------------------------------------------
% Die Pakete sind alle in der TeX Live Distribution enthalten. Wichtige Adressen
% www.ctan.org, www.dante.de

% Sprachunterstützung
\usepackage[ngerman]{babel}

% Benutzung von Umlauten direkt im Text
% entweder "latin1" oder "utf8"
\usepackage[utf8]{inputenc}

% Pakete mit Mathesymbolen und zur Beseitigung von Schwächen der Mathe-Umgebung
\usepackage{latexsym,exscale,stmaryrd,amssymb,amsmath}


\usepackage[nointegrals]{wasysym}
\usepackage{eurosym}

% Anderes Literaturverzeichnisformat
%\usepackage[square,sort&compress]{natbib}
\usepackage{hyperref}
% Für Farbe
\usepackage{color}
\usepackage{graphicx}
\usepackage{wrapfig}
\usepackage{subfigure}

% Caption neben Abbildung
\usepackage{sidecap}


% Befehl für "Entspricht"-Zeichen
\newcommand{\corresponds}{\ensuremath{\mathrel{\widehat{=}}}}
% Befehl für Errorfunction
\newcommand{\erf}[1]{\text{ erf}\ensuremath{\left( #1 \right)}}


%Fußnoten zwingend auf diese Seite setzen
\interfootnotelinepenalty=1000

%Für chemische Formeln (von www.dante.de)
%% Anpassung an LaTeX(2e) von Bernd Raichle
\makeatletter
\DeclareRobustCommand{\chemical}[1]{%
  {\(\m@th
   \edef\resetfontdimens{\noexpand\)%
       \fontdimen16\textfont2=\the\fontdimen16\textfont2
       \fontdimen17\textfont2=\the\fontdimen17\textfont2\relax}%
   \fontdimen16\textfont2=2.7pt \fontdimen17\textfont2=2.7pt
   \mathrm{#1}%
   \resetfontdimens}}
\makeatother
\usepackage{textcomp}
\usepackage{upgreek}
%\begin{document}
%$\upmu$
%\end{document}
%Honecker-Kasten mit $$\shadowbox{$xxxx$}$$
\usepackage{fancybox}

%SI-Package
\usepackage{siunitx}

%keine Einrückung, wenn Latex doppelte Leerzeile
\parindent0pt

%Bibliography \bibliography{literatur} und \cite{gerthsen}
%\usepackage{cite}
\usepackage{babelbib}
\selectbiblanguage{ngerman}

\usepackage{siunitx}
%\begin{document}
 % \SI{1.55}{\micro\metre}
\sisetup{math-micro=\text{µ},text-micro=µ}
\begin{document}

\begin{titlepage}
\centering
\textsc{\Large Praktikum zur Einführung in die physikalische Chemie,\\[1.5ex] Universität Göttingen}

\vspace*{2cm}

\rule{\textwidth}{1pt}\\[0.5cm]
{\huge \bfseries
  V3: Molmassenbestimmung\\[1.5ex]
  nach Viktor Meyer}\\[0.5cm]
\rule{\textwidth}{1pt}

\vspace*{1cm}


\begin{Large}
\begin{tabular}{ll}
Durchführende: &  Alea Tokita, Julia Stachowiak\\
Assistentin: & Annemarie Kehl\\
 Versuchsdatum: & 21.12.2015\\
 Datum der ersten Abgabe: & 11.01.2016\\
 Datum der zweiten Abgabe: & 25.01.2016      \\
 Datum der dritten Abgabe: & 04.02.2016\\
\end{tabular}
\end{Large}

\vspace*{2cm}

\begin{Large}
\fbox{
  \begin{minipage}[t][5cm][t]{7cm} 
  Messwerte:\\
   $M_1 = (86 \pm 2){~} \mathrm{g{~}mol^{-1}}$\\
   $M_2 = (88 \pm 2){~} \mathrm{g{~}mol^{-1}}$\\ 
\\
Literaturwert: \\
$M_\mathrm{Cl_2CH_2} = 84,93\, \mathrm{g{~}mol^{-1}}$
  
  
  
  
  \end{minipage}
}
\end{Large}

\end{titlepage}

\tableofcontents

\newpage
\section{Theorie}
In der Bestimmung der Molmasse nach Viktor Meyer wird die Molmasse einer leicht verdampfbaren Substanz mithilfe des von ihr verdrängten Gasvolumens beim Verdampfen bestimmt. Dabei handelt es sich um eine flüchtige Substanz, d.h. sie verdampft schon bei niedrigen Temperaturen unter $100{~}^{\circ}$C. In den Berechnungen wird näherungsweise von einem idealen Verhalten der Luft, sowie der verdampften Substanz und der Mischung beider ausgegangen. Es wird also angenommen, dass sich diese Gase wie ideale Gase verhalten, dh. keine Ausdehnung besitzen (Punktmassen)  und untereinander nur über elastische Stöße wechselwirken, sich aber gegenseitig nicht anziehen oder abstoßen. Dadurch lässt sich das ideale Gasgesetz auf sie anwenden.\\

Bei Mischung zweier idealer Gase gleichen Druckes und gleicher Temperatur addieren sich deren Volumina und die Gasmischung verhält sich somit ebenfalls ideal. Die Gase stoßen sich demnach weder ab oder ziehen sich an und besitzen Punktmassen.\\

Allgemein gilt bei idealen Gasmischungen für Stoffmengenanteil $x_i$ und den Partialdruck $p_i$ einer Komponente: 
\begin{align}
x_i = \frac{n_i}{\sum n_i}
\end{align}
und
\begin{align}
p_i = x_i \cdot p_{\mathrm{gesamt}}
\end{align}\\\\

Da der Partialdruck $p_i$ derjenige Druck ist, welcher vorherrschen würde, wenn sich die Komponente I allein im Gesamtvolumen befände, ist das partielle molare Volumen  $v_i$ der Komponente I gegeben durch:

\begin{align}
V_i = x_i \cdot V_{\mathrm{m}}= x_i \cdot \frac{ V_{\mathrm{gesamt}}} {n_{\mathrm{gesamt}}}
\end{align}\\\\

Wobei $ V_m $ für das molare Volumen und $n_{\mathrm{gesamt}}$ für die Gesamtstoffmenge der Gasmischung steht.\\\\

Durch Einsetzen des idealen Gasgesetzes, Gleichung $(4)$, ergibt sich Gleichung $(5)$:

\begin{align}
pV_m = RT
\end{align}

\begin{align}
p_i = x_i \cdot p = \frac{x_in_{\mathrm{gesamt}}RT}{V_{\mathrm{gesamt}}} = \frac{n_iRT}{V_{\mathrm{gesamt}}}
\end{align}


Somit gelten folgende Beziehungen  für die Partialdrücke der Propesubstanz P und der Luft L:
\begin{align}
P_{\mathrm{P}}V_{\mathrm{gesamt}} = n_{\mathrm{p}}RT
\end{align}

\begin{align}
P_{\mathrm{L}}V_{\mathrm{gesamt}} = n_{\mathrm{L}}RT
\end{align}

\begin{align}
p = p_{\mathrm{P}} + p_{\mathrm{L}} = \frac{ \left( n_{\mathrm{P}} + n_{\mathrm{L}} \right)RT}{V_{\mathrm{gesamt}}}
\end{align}

Im Versuch wird die Probesubstanz P bekannter Masse, jedoch unbekannter Stoffmenge verdampft und verdrängt dann genau die gleiche Stoffmenge Luft L, die sie selbst besitzt. Bei Zimmertemperatur wird dann durch Bestimmung von \textit{p} und \textit{V} die Stoffmenge an verdrängter Luft und damit auch die Stoffmenge der Probesubtanz ermittelt, wodurch sich die Molmasse berechnen lässt:  

\begin{align}
p_{\mathrm{L}} V_{\mathrm{L}} = n_{\mathrm{P}}RT = \frac{m_{\mathrm{P}}}{M_{\mathrm{P}}RT}
\end{align}

\begin{align}
M_{\mathrm{P}} = \frac{m_{\mathrm{P}}\cdot R \cdot T_{\mathrm{Z}}}{p_{\mathrm{L}} \cdot V_{\mathrm{L}}}
\end{align}

\section{Experimentelles}
\subsection{Versuchsaufbau}

\begin{figure} [h!]
\begin{center}
\includegraphics[width=12cm]{Versuchsaufbau.png} 
\end{center}
\caption {Versuchsaufbau}
\end{figure}


In Abbildung 1 ist die Apparatur zur Versuchsdurchführung skizziert. Die Probesubtanz wird in dem Verdampfungsgefäß verdampft. Dieses taucht in das zum Teil mit siedendem Wasser gefülltes Siedegefäß ein, welches wiederum in einem Heizpilz steht und so erhitzt wird. Die Temperatur im Siedegefäß sollte um ca. $20{~} ^{\circ}$C höher liegen als die Siedetemperatur der Probesubstanz. \\\\
Die Probesubstanz wird in einem Kölbchen in eine Glasöse in Schliffstück A gehängt. Durch einen Dorn im Schliffstück B kann der Hals des Kölbchens zerbrochen werden und die Probesubstanz verdampft im Verdampfungsgefäß. Der Dampf schiebt dann die im Gefäß befindliche Luft vor sich bis in das Messgefäß, wodurch das verdrängte Volumen bestimmt werden kann.

\subsection{Durchführung}
Die Masse des leeren Kölbchens wird auf $0,1 {~} \mathrm{mg}$ genau bestimmt und maximal $0,10{~}\mathrm{g}$ der Probesubstanz mithilfe einer Wasserstrahlpumpe in das Kölbchen gesaugt, welches durch eine grobe Wägung überprüft wird. Nach Zuschmelzen des gereinigten Kölbchenhalses wird durch ein dritte Wägung die tatsächlich aufgenommene Substanzmenge $m_{\mathrm{P}}$ genau ermittelt.\\\\
Nun wird die Apparatur an den Schliffen A und B, Kern und Hülse der Schliffe gesäubert und neu gefettet. Anschließend wird alles nach der Skizze beschrieben vorsichtig zusammengesetzt, dabei wird darauf geachtet, dass die Schliffe schlierenfrei sind. Das Eudiometerrohr wird mit Wasser gefüllt.\\\\
Das Wasser wird durch den Heizpilz zum Sieden gebracht und sobald keine Luftblasen mehr aus dem Verbindungsrohr entweichen wird das Verbindungsrohr unter das Eudiometerrohr geschoben. Darauf zerbricht man durch Drehung des Schliffstücks A den Hals des Kölbchens, sodass dieses in den heißen Teil des Verdampfungsgefäßes fällt. Sobald der Verdampfungsprozess abgeschlossen ist, wird das Volumen der Verdrängten Luft, sowie die Höhe des Wasserstandes bestimmt.\\\\
Der Versuch soll einmal wiederholt werden, wobei das Verdampfungsgefäß zwischen den Versuchen ausreichend zu belüften ist.\\
Des weiteren muss der Luftdruck $p_\mathrm{L}$, sowie die Luftfeuchtigkeit und die Zimmertemperatur $T_{\mathrm{Z}}$ bestimmt werden.     


\section{Auswertung}

Nach dem idealen Gasgesetz ist die Größe des verdrängten Volumens eines Gases abhängig von der Temperatur, dem Druck und der Stoffmenge. Letztere setzt sich aus der Molaren Masse und der Masse der Substanz zusammen. Daraus ergibt sich die molare Masse der Probe.

\begin{equation}
M_\mathrm{P} = \frac{m_\mathrm{P}}{n_\mathrm{P}} =\frac{m_\mathrm{P} \cdot R \cdot T_\mathrm{Z}}{p_\mathrm{L} \cdot V_\mathrm{L}}
\end{equation}

Die Masse der Probesubstanz und das verdrängte Volumen können einfach bestimmt werden, ebenso die Temperatur des Wassers $T_\mathrm{Z}$, durch welches die verdrängte Luft direkt durchtritt. \\
Zu Beginn der Messung ist der Druck $p_\mathrm{L}$ innerhalb der Apparatur gleich dem Luftdruck außerhalb und somit leicht zu bestimmen. Bei Durchtritt der verdrängten Luft durch das Wasser erhöht sich allerdings auch die Luftfeuchtigkeit. Außerdem übt auch die Gasphase der Wassersäule einen hydrostatischen Druck aus, sodass die Luft nach Durchtritt durch das Wasser einen höheren Druck ausübt. \\
Der hydrostatische Druck ist mithilfe der barometrischen Höhenformel zu beschreiben:\\

\begin{equation}
p_\mathrm{hyd}= \rho_\mathrm{H_2O} \cdot g \cdot h
\end{equation}

Der Wasserdampfpartialdruck ist proportional zu der relativen Luftfeuchtigkeit $r$ und dem Dampfdruck des Wassers bei gegebener Temperatur. \\

\begin{equation}
p_{rel} = r\cdot p_\mathrm{H_2O}(T_\mathrm{Z})
\end{equation}

Nach Durchtreten der Luft durch das Wasser ist der Wasserdampfdruck gesättigt, sodass sich für $r = 1$ ergibt. Der Druck ist nun höher als zuvor und muss somit um den neu hinzugekommenen Wert $1- r$ korrigiert werden. \\
Das von der enstehenden Gasphase der Substanz verdrängte Volumen ist gleich dem aufgefangenen Volumen an Luft im Eudiometerrohr. Der Druck im Eudiometerrohr ist durch die mit Wasser gesättigte Luft allerdings größer. Somit muss der Atmosphärendruck $p_\mathrm{B}$ um die beiden Korrekturfaktoren reduziert werden. 

\begin{equation}
p_L = p_\mathrm{B} - p_\mathrm{hyd} - ([1-r] \cdot p_\mathrm{H_2O}(T_\mathrm{Z}))
\end{equation}

Der Druck und die Dichte des Wassers bei herrschender Temperatur werden einer Tabelle entnommen. Der vom Quecksilberbarometer angezeigte Athmosphärendruck wird mithilfe einer Korrekturtabelle in Torr und anschließend Pascal umgerechnet.\\
Folgende Messwerte sind für beide Rechnungen gleich:\\

\begin{table} [h]
\begin{tabular} {p {3 cm} p {5 cm}}
$T_\mathrm{Z} $ & $=21^\circ\text{C}$\\
	$ \rho_\mathrm{H_2O} $ & $=997,95\, \mathrm{kg} \cdot 				\mathrm{m^{-3}}$\protect\footnotemark\\
	$ r $ & $=0,54$\\
\end{tabular}
\end{table}

\footnotetext{http://webbook.nist.gov/chemistry/, Stand: 14.07.2011}

Außerdem wurde noch folgendes gemessen:\\

\begin{table} [h]
\centering
\begin{tabular}{|p{4 cm}||p{4 cm}|p{4 cm}|}
        \hline
		Werte & Messung 1 & Messung 2\\
         \hline 
        $ h$ in m & $0,369$  & $0,379$ \\
        \hline
        $ m$ in g & $0,0777$  & $0,0794$ \\
        \hline
        $ V $ in ml & $23,2$ & $23,2$ \\
        \hline     
\end{tabular}
\end{table}

Daraus ergeben sich durch einsetzen in Formel 11, 12 und 14 folgende Werte:


\begin{table} [h]
\centering
\begin{tabular}{|p{4 cm}||p{4 cm}|p{4 cm}|}
        \hline
		Werte & Messung 1 & Messung 2\\
         \hline 
        $ p_\mathrm{hyd}$ in Pa & $3612 \approx 361 \cdot 10$  & $3710 \approx 371 \cdot 10$ \\
        \hline
        $ p_\mathrm{L}   $ in Pa & $94838 \approx 948 \cdot 10^2$  & $94936 \approx 949 \cdot 10^2$ \\
        \hline
        $ M $ in $\mathrm{g{~}mol^{-1}}$ & $86$ & $88$ \\
        \hline     
\end{tabular}
\end{table}

\newpage
\section{Fehlerrechnung}

\subsection{absolute Fehler}

Die absoluten Fehler der Messgrößen ergeben sich aus der letzten Dezimalstelle. Der absolute Fehler des Athmosphärendrucks $p_\mathrm{B}$ errechnet sich aus dem Fehler der Kuppenhöhe und dem der Temperatur bei der Umrechnung in Torr. Die Fehler werden unten dargestellt.\\

\begin{table} [h]
\begin{tabular} {p {3 cm} p {5 cm}}
	$\Delta m $ & $=0,001$ g\\
	$\Delta p_\mathrm{B}  $ & $=0,07\, \mathrm{Torr} \approx 9$ Pa\\
	$\Delta \rho_\mathrm{H_2O} $ & $=0,01\, \mathrm{kg} \cdot 				\mathrm{m^{-3}}$\\
	$\Delta r $ & $=0,01$\\
	$\Delta V $ & $= 0,1 \cdot 10^{-6} \mathrm{m^3}$\\
	$\Delta p_\mathrm{H_2O}(T_\mathrm{Z}) $ & $= 0,1$ Pa\\
	$\Delta h $ & $ =0,005$ m\\

\end{tabular}
\end{table}

\subsection{Fehlerfortpflanzung}

Nach der Gauß'schen Fehlerfortpflanzung

\begin{equation}
\Delta f = \sqrt{\sum_\mathrm{i} \left(\frac{d}{dx_\mathrm{i}}\right)^{2} \cdot \Delta x_\mathrm{i}^2}
\end{equation}


ergibt sich für den hydrostatischen Druck $\Delta p_\mathrm{hyd}$ folgender Fehler:\\

\begin{equation}
\Delta p_\mathrm{hyd} =  \sqrt{(g \cdot h)^2 \cdot \Delta \rho_\mathrm{H_2O}^2 + (\rho_\mathrm{H_2O} \cdot g)^2 \cdot \Delta h^2}
\end{equation}


Anschließend ergibt sich für den Fehler des Luftdruckes $\Delta p_\mathrm{L}$:\\


\begin{equation}
\Delta p_\mathrm{L} = \sqrt{\Delta p_\mathrm{B}^2 + \Delta p_\mathrm{hyd}^2 + p_\mathrm{H_2O}(T_\mathrm{Z})^2 \cdot \Delta r^2 + (1-r)^2 \cdot \Delta p_\mathrm{H_2O}(T_\mathrm{Z})}
\end{equation}

Und anschließend für den Fehler der Molaren Masse:\\

\begin{equation}
\Delta M = \sqrt{\left(\frac{R\cdot T}{p\cdot V}\right)^2 \cdot \Delta m^2 + \left(\frac{m\cdot R}{p\cdot V} \right)^2 \cdot \Delta T^2 + \left(-\frac{m\cdot R\cdot T}{p^2 \cdot V}\right)^2 \cdot \Delta p^2 + \left(-\frac{m \cdot R \cdot T}{p \cdot V^2} \right)^2 \cdot \Delta V^2}
\end{equation}

Mit $p = p_\mathrm{L}$ ergeben sich damit folgende Fehler:\\

\begin{table} [h]
\centering
\begin{tabular}{|p{4 cm}||p{4 cm}|p{4 cm}|}
        \hline
		Fehler & Messung 1 & Messung 2\\
         \hline 
        $\Delta p_\mathrm{hyd}$ in Pa & $48,95 \approx 50$  & $48,95 \approx 50$ \\
        \hline
        $\Delta p_\mathrm{L}   $ in Pa & $1344 \approx 1 \cdot 10^3$  & $1344 \approx 1 \cdot 10^3$ \\
        \hline
        $\Delta M $ in $\mathrm{g{~}mol^{-1}}$ & $1,72 \approx 2$ & $1,74 \approx 2$ \\
        \hline     
\end{tabular}
\end{table}




\subsection{Diskussion systematischer Fehler}

Zur Errechnung der molaren Masse wird das ideale Gasgesetz verwandt und somit davon ausgegangen, dass sich die verdampfte Substanz und ebenso auch die Luft wie ein ideales Gas verhalten. \\

In Realität verhält es sich jedoch anders, wodurch es je nach Abstoßung oder Anziehung des Gases der Substanz zu einer positiven oder negativen Abweichung kommen kann. Da es sich um verschiedene Gase handelt (Gemisch mit der Luft), können sich die Wechselwirkungen gegenseitig aufheben oder addieren.\\ 

Ein weiterer systematischer Fehler ist die nicht konstante Temperatur des Wasserbades. Durch den hindurchtretenden Wasserdampf erwärmt sich das Wasserbad und die der Tabelle entnommene Dichte des Wassers ist somit auch nicht konstant sondern wird während der Messung größer. Dadurch wird der hydrostatische Druck größer und der Luftdruck kleiner, was eine positive Abweichung der molaren Masse zur Folge hat.\\


Eine weitere Schwierigkeit stellt die Höhenmessung dar. Sie erfolgte mittels eines Zollstockes und muss im rechten Winkel zum Boden des Wasserbades gemessen werden, was mittels Zollstock schwer zu verwirklichen ist. Die Höhe wurde durch falsches Anlegen des Zosstockes somit wahrscheinlich größer bestimmt, was den hydrostatischen Druck ebenfalls größer und dadurch den  Luftdruck kleiner macht. Dies hat ebenfalls eine positive Abweichung des Literaturwertes im Bezug auf die Messwerte zur Folge.
\\




\section{Diskussion}

\subsection{Vergleich mit Literaturwerten}
Der Literaturwert von Dichlormethan liegt bei 
$M_\mathrm{S} = 84,93\, \mathrm{g\,mol^{-1}}$\protect\footnotemark
\\

\footnotetext{http://www.chemie.de/lexikon/Dichlormethan.html, abgerufen am \textbf{24.01.16} }



Aus den Messungen ergeben sich folgende Werte:\\


\begin{table} [h]
\centering
\begin{tabular}{|p{4 cm}||p{4 cm}|p{4 cm}|}
        \hline
		Werte & Messung 1 & Messung 2\\
         \hline 
        $ p_\mathrm{hyd}$ in Pa & $3612 \pm 50$  & $3710 \pm 50$ \\
        \hline
        $ M $ in $\mathrm{g{~}mol^{-1}}$ & $86 \pm 2$ & $88 \pm 2$ \\
        \hline     
\end{tabular}
\end{table}

\subsection{Diskussion}


Die gemessenen Werte liegen sehr nah an den Literaturwerten, wodurch sich zeigt, dass die Messung sehr genau ist, oder sich Fehler gegenseitig ausgleichen. Dennoch schließen die Fehlergrenzen der zweiten Messung den Literaturwert nicht ein; es muss also noch systematische Fehler geben. \\
Wie im vorigen Abschnitt dargestellt, verursachen Fehler in der Höhen- und Temperaturmessung eine Abweichung der gemessenen Werte unterhalb des Literaturwertes, tatsächlich sind die gemessenen Werte allerdings größer. Dies lässt darauf schließen, dass der systematische Fehler der Näherung das Gas verhalte sich ideal, einen größeren systematischen Fehler darstellt und die anderen beiden Fehler sogar teilweise kompensiert, da der Fehler genau in die andere Richtung wirkt.\\

Es gibt sehr viele fehlerbehaftete Größen im Experiment, wodurch sich Messfehler auch über die Fehlergrenzen hinaus fortpflanzen können, wenn die angenommenen Gerätefehler zu klein gewählt wurden.\\

 Außerdem ist es auch nicht möglich, anhand der Messwerte einzuschätzen, ob das Gas der Substanz positiv oder negativ vom idealen Verhalten abweicht. Die zu große ermittelte molare Masse müsste von einem zu kleinen Volumen kommen, was eine negative Abweichung vom idealen Verhalten zur Folge hätte; diese Aussage kann allerdings aufgrund der zu hohen Ungenauigkeit und der zu kleinen Abweichung nicht getroffen werden.\\

Für genauere Werte müsste mit einer Virialentwicklung gerechnet werden, die Abweichung von 2 bzw. 4,5 Prozent zeigt jedoch, dass in guter Näherung mit dem idealen Gasgesetz gerechnet werden kann.

\newpage

\section{Literaturverzeichnis}
\begin{flushleft}
1 \quad Gerd Wedler: \emph{Lehrbuch der physikalischen Chemie}, 5. Aufl., WILEY-VCH Verlag GmbH Co. KGaA, Weinheim, \textbf{2004}.\\
\vspace{0,5 cm}
2\quad Götz, Eckold: \emph{Sriptum zur Einführung in die physikalische Chemie}, Institut für physikalische Chemie, Uni Göttingen, \textbf{2015}.\\
\vspace{0,5 cm}
3 \quad \emph{Skriptum für das Praktikum zur Einführung in die Physikalische Chemie}, Institut für physikalische Chemie, Uni Göttingen, \textbf{2015}.\\
\end{flushleft}



\end{document}
